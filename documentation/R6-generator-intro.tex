% !TeX root = RJwrapper.tex
\title{R6 Generator Maven Plugin: Bridging Java and R6 for Package Development}
\author{by Robert Challen}

\maketitle

\abstract{%
This article introduces a novel approach to integrating Java code into
R, that offers distinct advantages in terms of ease of development and
maintenance to Java programmers wishing to expose their libraries as R
packages. It builds on the low level interface provided by the
\CRANpkg{rJava} package, Java code analysis tools, and the Maven
compiler framework, to programmatically generate R package code. The R6
Generator Maven Plugin provides a Java annotation processor and R code
generator that creates an R package exposing an \CRANpkg{R6} based
interface to Java code. By working at compile time it ensures the
bi-directional transfer of data between R and Java is type-safe and
fast, it allows a simple route to maintainence of R ports of Java
libraries, and it automates package documentation.
}

\hypertarget{introduction}{%
\section{Introduction}\label{introduction}}

The R6 Generator Maven Plugin provides a low friction route to
integrating Java code into R programs. \CRANpkg{rJava} is a low level
interface for using Java libraries in R via the Java Native Interface
(JNI). However use of this low level interface has a steep learning
curve for both R and Java programmers, as it requires an in-depth
understanding of the limitations of the JNI interface between R and
Java. The \CRANpkg{jsr223} package aims to made to reduce this friction
in the situation where dynamic interaction between R and Java is needed,
by integration of Java based scripting languages, such as Kotlin or
Groovy, and a ancilliary package that manages dynamic data-type
translation - \CRANpkg{jdx}. Although simplifying data type conversion
this is still a relatively complex solution for the situation where a
developer wishes to package Java based functions for use in R programs,
for native R use as a library.

Strengths and weaknesses of Java and R R ease of data import and
integration, spatial support R visualisation R data wrangling R REPL
Java ease of multithreaded processing. Java dependency and classpath
management. Java VM debugging versus R debugging

The existing approaches - Integration between Java and R has existed
since the release of \CRANpkg{rJava} in 2003, however successful use of
Java libraries within R has remained complex and

rJava, rjsr223 jsr223 reference - review of Java integration JNI \&
rJava features and limitations rJava 2 level api jsr223 benefits and
limitations Data only integration e.g.~Apache Arrow.

\hypertarget{use-cases}{%
\section{Use cases}\label{use-cases}}

\begin{itemize}
\tightlist
\item
  Java library development for R use
\item
  Adaption of java library for R use
\end{itemize}

\hypertarget{desiderata-design-rationale}{%
\section{Desiderata / Design
rationale}\label{desiderata-design-rationale}}

\begin{itemize}
\tightlist
\item
  R as REPL, java as backend
\item
  No knowledge of JNI, rJava or intermediate languages required.
\item
  High performance - minimise interpretation, Compile time - strongly
  typed inputs to R - minimise data transfer overhead. minimise use of
  reflection
\item
  Accurate data transformations and round trip
\item
  Dependency management -
\item
  Separation of concerns \& isolation - java based API layer to isolate
  Java changes from R API changes
\item
  Predictability - Runtime library - predictable R data in Java code
  isolating java from R type system
\item
  Seamless R use of library - R6 class hierarchy - native R facade to
  Java code
\item
  CRAN ready submission from Java library
\end{itemize}

\hypertarget{terminology-and-concepts}{%
\section{Terminology and concepts}\label{terminology-and-concepts}}

Maven plugin - Runtime dependency - R types for java

\hypertarget{feature-documentation}{%
\section{Feature documentation}\label{feature-documentation}}

\hypertarget{minimal-example}{%
\subsection{Minimal example}\label{minimal-example}}

\begin{minted}[linenos]{java}
import org.slf4j.Logger;
import org.slf4j.LoggerFactory;

import uk.co.terminological.rjava.RClass;
import uk.co.terminological.rjava.RMethod;
import uk.co.terminological.rjava.types.RDataframe;

/**
 * This class is a very basic example of the features of the rJava maven plugin. <br/>
 * The class is annotated with an @RClass to identify it as part of the R API. <br/>
 */
@RClass
public class MinimalExample {

    static Logger log = LoggerFactory.getLogger(MinimalExample.class);
    
    @RMethod(examples = {
        "minExample = J$MinimalExample$new()",
        "minExample$demo(dataframe=tibble::tibble(input=c(1,2,3)), message='Hello world')"
    })
    /**
     * Documentation of the method can be done in JavaDoc and these will be present in the R documentation 
     * @param dataframe - a dataframe with an arbitrary number of columns
     * @param message - a message
     * @return the dataframe unchanged
     *  
     */
    public RDataframe demo(RDataframe dataframe, String message) {
        log.info("this dataframe has nrow="+dataframe.nrow());
        log.info(message);
        return dataframe;
    }
    
}
\end{minted}

\begin{minted}[linenos]{xml}
    <properties>
        <project.build.sourceEncoding>UTF-8</project.build.sourceEncoding>
        <maven.compiler.source>1.8</maven.compiler.source>
        <maven.compiler.target>1.8</maven.compiler.target>
        <r6.version>master-SNAPSHOT</r6.version>
    </properties>
...
    <dependencies>
        <dependency>
            <groupId>com.github.terminological</groupId>
            <artifactId>r6-generator-runtime</artifactId>
            <version>${r6.version}</version>
        </dependency>
    </dependencies>
...
    <!-- Resolve runtime library on github -->
    <repositories>
        <repository>
            <id>jitpack.io</id>
            <url>https://jitpack.io</url>
        </repository>
    </repositories>

    <!-- Resolve maven plugin on github -->
    <pluginRepositories>
        <pluginRepository>
            <id>jitpack.io</id>
            <url>https://jitpack.io</url>
        </pluginRepository>
    </pluginRepositories>
...
            <plugin>
                <groupId>com.github.terminological</groupId>
                <artifactId>r6-generator-maven-plugin</artifactId>
                <version>${r6.version}</version>
                <configuration>
                    <packageData>
                        <title>A test library</title>
                        <version>0.01</version>
<!--                        <debug>true</debug> -->
                        <rjavaOpts>
                            <rjavaOpt>-Xmx256M</rjavaOpt>
                        </rjavaOpts>
                        <packageName>testRapi</packageName>
                        <license>MIT</license>
                        <description>A long description of the package.</description>
                        <maintainerName>test forename</maintainerName>
                        <maintainerFamilyName>optional-surname</maintainerFamilyName>
                        <maintainerEmail>test@example.com</maintainerEmail>
                        <maintainerOrganisation>University of Examples</maintainerOrganisation>
                    </packageData>
                    <outputDirectory>${project.basedir}/r-library</outputDirectory>
                </configuration>
                <executions>
                    <execution>
                        <id>generate-r-library</id>
                        <goals>
                            <goal>generate-r-library</goal>
                        </goals>
                    </execution>
                </executions>
\end{minted}

\hypertarget{r6-class-generation}{%
\subsection{R6 class generation}\label{r6-class-generation}}

\hypertarget{maven-packaging}{%
\subsection{Maven packaging}\label{maven-packaging}}

\hypertarget{generated-r6-documentation}{%
\subsection{Generated R6
documentation}\label{generated-r6-documentation}}

\hypertarget{datatype-transformation}{%
\subsection{Datatype transformation}\label{datatype-transformation}}

\hypertarget{java-to-r}{%
\subsubsection{Java to R}\label{java-to-r}}

\hypertarget{r-to-java---runtime-library}{%
\subsubsection{R to Java - runtime
library}\label{r-to-java---runtime-library}}

\hypertarget{debugging-java-code}{%
\subsection{Debugging Java code}\label{debugging-java-code}}

\begin{itemize}
\tightlist
\item
  Datasets for testing Java code / serialisation of inputs
\item
  Debugging flag
\end{itemize}

\hypertarget{benefits}{%
\section{Benefits}\label{benefits}}

\begin{itemize}
\tightlist
\item
  Low friction
\item
  Minimal dependencies introduced.
\item
  Java Library dependency packaging
\item
  Accurate bidirectional transfer of data between R and Java
\item
  Type safety
\item
  Code completion and type hinting in R
\item
  Fluent use of R data in java
\item
  R Documentation
\item
  Separation of concerns
\item
  Maintenance
\end{itemize}

\hypertarget{limitations}{%
\section{Limitations}\label{limitations}}

\begin{itemize}
\tightlist
\item
  Fat Jar bloat
\item
  Recompilation and iterative development
\item
  Naming collisions
\item
  Concurrent use of Multiple rJava libraries
\item
  Java class caching
\end{itemize}

\hypertarget{future-development}{%
\section{Future development}\label{future-development}}

\begin{itemize}
\tightlist
\item
  Tighter integration of multithreading and promises.
\item
  Matrix support
\item
  Named rows in dataframes
\item
  R test case generation
\item
  Java object bindings
\item
  Purrr style lists in data frames
\end{itemize}

\hypertarget{introduction-1}{%
\section{Introduction}\label{introduction-1}}

\bibliography{R6-generator-intro.bib}

\address{%
Robert Challen\\
University of Exeter\\%
EPSRC Centre for Predictive Modelling in Healthcare,\\ University of Exeter,\\ Exeter,\\ Devon, UK.\\
%
\url{https://github.com/terminological/r6-generator-maven-plugin}%
\\\textit{ORCiD: \href{https://orcid.org/0000-0002-5504-7768}{0000-0002-5504-7768}}%
\\\href{mailto:rc538@exeter.ac.uk}{\nolinkurl{rc538@exeter.ac.uk}}
}

